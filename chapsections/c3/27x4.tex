No. They wouldn’t leave the three-dial padlocks unsecured out of trust alone. These things must be so cheap and ineffective that they’d only inconvenience a dedicated thief.\\

You hold one in your hand... and find that it feels all too familiar. Your hands begin to move on their own, yanking down on the lock to put pressure on the mechanism within. Then you spin the first dial, and test the second’s spin against each number. Its spin feels sticky when the first dial is set to ‘6’. That’s one down. You repeat the process for the second and third dials, and find that when the second dial is set to ‘3’ the third begins to stick. From there, it’s only a simple process of elimination to find out the third dial’s number: ‘7’.\\
\gain{Effigy}\\
\gain{Copper Schilling} x5\\
\gain{Silver Schilling}\\

A poorly-made lock, but it looks like the local dice champion had no other choice.\\

>> Return to the barracks -- \turnto{c327}