You pass through the barrier, and enter into a curious scene. Two columns of hollowed prisoners stand here amongst the fog, chained together by their ankles. In their hands, they bear an assortment of makeshift armaments: flagpoles, stockades, and railings filed into crude spears, and shields fashioned from hacked-apart mess tables.\\
The prisoners waver idly in their formation, adrift in their own maddened thoughts. A revolt forgotten in its infancy by both guard and revolutionary alike. Yet their heads pick up at the sound of your approaching footfalls. And the phalanx wheels in your direction, its rows of ramshackle spears dropping into a defensive posture.

\subsection*{Victory Condition}
Defeat all enemies

\subsection*{Doom Events}
For this encounter, \emph{doom events} are triggered by the remaining number of enemies, not \emph{round tallies}.

\begin{itemize}
\item \textbf{10 Remaining:} \emph{The phalanx begins to march--lurching, and badly out of step.} Begin the encounter by rolling on table A.
\item \textbf{8 Remaining:} \emph{The Lost Phalanx drags its dead members along by their chains.} The Lost Phalanx loses 1 Move value.
\item \textbf{6 Remaining:} \emph{The number of dead members has severely slowed the formation.} The Lost Phalanx loses 1 Move value.
\item \textbf{4 Remaining:} \emph{The remaining hollows are incapable of movement.} The last 4 Hollowed Prisoner Phalanx entities are no longer part of The Lost Phalanx formation, and act as individual entities. Roll on Encounter Table B for this and all future Rounds.
\end{itemize}

\pagebreak

\subsection*{Formation Fighter Rules}
\textbf{The Formation} -- The Lost Phalanx must maintain its original “two columns” formation as best as possible, though it may snake and bend. See the example images below.\\

\textbf{Movement} -- During the \emph{counterbeat}, the entire formation moves before its individual entities commit any attacks. Move the formation by first moving its front row, and then fill in the rest of the phalanx behind it. The “front” is whichever row is closest to the character at the start of the Turn. If the character is positioned off to one side of the phalanx, so that multiple rows are equidistant, then then move the formation via the nearest column instead (the formation will move sideways).

\begin{tcolorbox}
\textbf{Note:} Remember, only the front row/column moves towards the character. The rest of the phalanx just fills in behind the front, maintaining the original formation as best as possible.
\end{tcolorbox}

\textbf{Attacking} -- Resolve each attack for all individual entities, starting sequentially from the front. Within each row/column, the player may resolve the individual entities’ Turns in any order of their choosing. When an encounter roll has multiple attacks, resolve all attacks of the first type before moving onto the second. Entities in the formation may attack past each other. \emph{Each individual entity may only resolve one attack per Turn.}\\

\textbf{Damage} --  Treat The Lost Phalanx as a single entity with one \textbf{HP} pool. For every 3 damage tokens assigned to \textbf{HP}, remove the last damaged entity/entities. The formation will reform its ranks next time it moves. Area of Effect attacks (such as firebombs) will inflict damage for each individual entity struck.

\subsection*{Example Images}
\begin{center}
\framebox{\includegraphics[width = 0.3\textwidth]{./maps/c214a.png}}
\framebox{\includegraphics[width = 0.3\textwidth]{./maps/c214b.png}}
\framebox{\includegraphics[width = 0.3\textwidth]{./maps/c214c.png}}
\\
\emph{From left to right:\\
(1) The formation (red) moves towards the character (goldenrod) during its Turn.\\
(2) The Move begins with the members of the first row (who keep 1 tile distance).\\
(3) Then the rest of the phalanx fills in, maintaining the formation as best as possible.}\\
\ \\
\framebox{\includegraphics[width = 0.3\textwidth]{./maps/c214d.png}}
\framebox{\includegraphics[width = 0.3\textwidth]{./maps/c214e.png}}
\framebox{\includegraphics[width = 0.3\textwidth]{./maps/c214f.png}}
\\
\emph{The formation moves sideways via the nearest column if the character stands off to one side.}
\end{center}

\pagebreak

\subsection*{Encounter Tables}
\begin{tcolorbox}
\textbf{A - The Lost Phalanx}\\
\textbf{Roll:} 2D6
\begin{center}
\begin{tabular}{ L | L | L }
\multicolumn{1}{c|}{\textbf{2}} & 
\multicolumn{1}{c|}{\textbf{3}} & 
\multicolumn{1}{c}{\textbf{4-5}} \\
\emph{Charge!} &
\emph{Thrown Debris} &
\textbf{A:} \emph{Shield Push. Makeshift Spear}\newline \textbf{B:} Move. \emph{Makeshift Spear} \\
\hline
\multicolumn{1}{c|}{\textbf{6}} & 
\multicolumn{1}{c|}{\textbf{7}} & 
\multicolumn{1}{c}{\textbf{8}} \\
\emph{Shield Push. Makeshift Spear} &
\textbf{A:} \emph{Shield Push. Makeshift Spear}\newline \textbf{B:} Move. \emph{Makeshift Spear} &
\emph{Shield Push. Makeshift Spear} \\
\hline
\multicolumn{1}{c|}{\textbf{9-10}} & 
\multicolumn{1}{c|}{\textbf{11}} & 
\multicolumn{1}{c}{\textbf{12}} \\
\textbf{A:} \emph{Shield Push. Makeshift Spear}\newline \textbf{B:} Move. \emph{Makeshift Spear} &
\emph{Thrown Debris} &
\emph{Charge!}
\end{tabular}
\end{center}
\end{tcolorbox}

\begin{tcolorbox}
\textbf{B - Hollowed Prisoner Phalanxes}\\
\textbf{Roll:} 1D6
\begin{center}
\begin{tabular}{ L | L | L }
\multicolumn{1}{c|}{\textbf{1}} & 
\multicolumn{1}{c|}{\textbf{2-5}} & 
\multicolumn{1}{c}{\textbf{6}} \\
\textbf{A:} \emph{Shield Bash}\newline \textbf{B:} \emph{Thrown Debris} &
\textbf{A:} \emph{Shield Up!}\newline \textbf{B:} \emph{Makeshift Spear}\newline \emph{This result is only exhausted after its third token} &
\textbf{A:} \emph{Shield Bash}\newline \textbf{B:} \emph{Thrown Debris}
\end{tabular}
\end{center}
\textbf{Note:} Resolve the enemies’ Turns in any order of the player’s choosing.
\end{tcolorbox}

\pagebreak

\subsection*{Enemy Sheets}
\hrule
\ \\
{\large \textbf{The Lost Phalanx}}\\\\
\begin{tabular}{s s s}
\textbf{HP:} N/A & \multicolumn{2}{l}{\textbf{Move:} 3, keeps 1 tile distance}\\
\end{tabular}\\

\emph{Formation Fighter -- Hollowed Prisoner Phalanx:} This entity consists of multiple, individual entities. Remove an entity for every 3 damage assigned to \textbf{HP}. See the Formation Fighter Rules section on the previous page for more details.\\

\emph{Shieldwall:} All entities in this formation always possess \emph{Shield Up!} and have unlimited shield Stability.\\

\textbf{Attacks:}
\begin{itemize}
\item \emph{Charge!} - Move The Lost Phalanx directly forward (via the front row/column) in a straight \& snaking line until any member attempts to Move onto an impassable tile, map boundary, or the character’s tile. In that case, end the Move for the entire formation. If the character’s tile is encountered in this manner, inflict 2 \emph{Unparryable} and \emph{Undodgeable} Crush damage, Knockback 2, and Knockdown on the character. \emph{Always resolve this attack, regardless of whether it will strike the character.}
\end{itemize}
\hrule
\ \\
\begin{tcolorbox}
\textbf{Note:} Because of \emph{Shieldwall}, there’s no need to track \emph{Shield Up!}, or shield Stability or Durability, for individual members of The Lost Phalanx. Just reduce all incoming damage by 1.
\end{tcolorbox}
\begin{tcolorbox}
\textbf{Note:} \emph{Charge!} is \emph{Undodgeable}, but the character might still be able to move out of the way via a reaction.
\end{tcolorbox}
\ \\
\hrule
\ \\
{\large \textbf{Hollowed Prisoner Phalanx}}\\\\
\begin{tabular}{s s s}
\textbf{HP:} 3 & \multicolumn{2}{l}{\textbf{Move:} 0}\\
\textbf{Shield Def:} 1 & \textbf{Shield Stab:} 5 & \textbf{Shield Dur:} 1\\
\end{tabular}\\

\emph{Hollow:} This entity ignores the Charmed, Maddened, and Fear conditions.\\

\textbf{Attacks:}
\begin{itemize}
\item \emph{Makeshift Spear} - Deal 1 Pierce damage to an entity within 2 tiles. Do not remove \emph{Shield Up!}
\item \emph{Shield Push} - Inflict Knockback 1 to an adjacent entity. Do not remove \emph{Shield Up!}
\item \emph{Shield Bash} - Deal 1 Crush damage and Knockback 1 to an adjacent entity. Do not remove \emph{Shield Up!}
\item \emph{Thrown Debris} - Deal 1  ranged Crush damage to an entity that is within 2-3 tiles.
\end{itemize}
\hrule
\ \\

\begin{tcolorbox}
\textbf{Note:} If entities are not part of The Lost Phalanx formation, they no longer benefit from \emph{Shieldwall}.
\end{tcolorbox}

\pagebreak

\subsection*{Encounter Map}
\begin{center}
\framebox{
\includegraphics[width = 0.96\textwidth]{./maps/c214.png}
}
\end{center}

\subsection*{Setup Instructions}
\begin{itemize}
\item \textbf{Goldenrod:} Character Start Location. Place the character on either tile.
\item \textbf{Red:} Enemy Start Locations.
\item \textbf{Black:} Full-Cover.
\item \textbf{Gray:} Half-Cover.
\item \textbf{Green:} Escape Tile.
\end{itemize}

\pagebreak

\subsection*{Victory}
The last of the phalanx falls, his makeshift spear clattering on the flagstone. Despite its crude construction, the weapon was made with ingenuity and great care. Its spearhead, fashioned from the shard of a chamber pot, is secured to the wooden shaft with a rag torn from the man’s own shirt. Looking over the rest of your former comrades, you discover that he didn’t make that alteration alone. The entire phalanx was dressed in a sort of tatterdemalion uniform: their shirts all torn across the chest, displaying the Sign with brazen pride. Everloyal.\\

>> Souls of a Doomed Rebellion (20)\\
\gainx{Warcry}\\
\gain{Makeshift Spear}\\
\gain{Table Shield}\\
\notegain{c214a} The Lost Phalanx is defeated\\
>> \turnto{c215}

\subsection*{Defeat}
A spearhead pierces your flesh, and then another. You stagger backwards, blood pulsing from your wounds like the bungholes of a cask. Your former comrades were damnably well-prepared for this. Even in the throes of madness, they manage to keep up their shieldwall.\\

You have no choice but to flee: scrambling back to fog barrier under a hail of thrown debris. You’ll survive the experience, but at the cost of something inside you.\\

>> Clear all \textbf{HP} slots\\
>> \textbf{HMN} - 1\\
>> \turnto{c213}

\subsection*{Retreat}
You flee back through the fog barrier, leaving your former comrades to march their formation aimlessly about the cell block.\\

>> Clear all \textbf{HP} slots\\
>> \turnto{c213}\\

\begin{tcolorbox}
\textbf{Note:} Upon defeat or retreat, you may repeat this encounter as many times as necessary. However, the encounter will begin with The Lost Phalanx at full strength again.
\end{tcolorbox}