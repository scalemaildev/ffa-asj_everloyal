\subsection{Mystic Keys}
Something of a mania amongst the followers of Av*rn-Z*l, the Mystic Keys is a collection of secret words, mantras, and parables that produce miraculous effects when spoken--so long as the speaker is sufficiently convinced of their truthfulness. The Keys (as it’s often called in short) is an unwritten codex, and King Z*l’s faithful are forbidden from translating any of its contents into writing upon pain of death. Despite this, there is still some iconography within the religion, the most popular of which being the Sign: a thin, hollow circle representing loyalty.

\begin{itemize}
\item \textbf{Catalyst:} Icons (Optionally)
\item \textbf{Bonus:} \textbf{LB}
\item \textbf{Notes:} “Mantras” are \emph{static} conditions. The status sheet is limited to 1 Mantra token at a time, which is replaced when another mantra is activated. See the Conditions Roster section for details
\end{itemize}

\begin{center}
\begin{tabularx}{\textwidth}{p{0.1\textwidth}p{0.1\textwidth}p{0.05\textwidth}p{0.05\textwidth}p{0.1\textwidth}p{0.46\textwidth}}
\hline
\rowcolor{white} \textbf{Name} & \textbf{Req} & \textbf{Int} & \textbf{FP} & \textbf{SP} & \textbf{Effects}\setcounter{rownum}{0}\\
\hline
\makeitem{Mantra: Dedication} & FTH: 10 & 1 & 1 & 1D & Immediately after rolling the \emph{stamina pool}, the character may re-roll \textbf{SP} dice for 3 \textbf{FP} per die. Remember that re-rolls are once-per-die and the second result is final \\
\makeitem{Hope} & FTH: 10 & 1 & 2 & 1D & Summons a light source until the next \emph{short reprieve}.\newline If cast during an encounter, the character gains \textbf{PWR} tokens of Enhanced Defense: \textbf{D.DEF} (1) \\
\makeitem{Masin Crosses the River} & FTH: 12 & 1 & 2 & 4 & Move 5, ignoring the effects of any hazard tiles \\
\makeitem{Praise} & FTH: 12 & 2 & 4 & 3 & Remove Bleeding and any unassigned damage tokens from the character sheet \\
\makeitem{Succor} & FTH: 10 & 1 & 3 & 4 & Remove \textbf{PWR} damage tokens from the character’s \textbf{HP} slots \\
\hline
\end{tabularx}
\end{center}