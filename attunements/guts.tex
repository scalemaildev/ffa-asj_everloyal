\subsection{Guts}
Guts are fighting techniques and stances that can give a warrior a much-needed edge in combat.
\begin{itemize}
\item \textbf{Catalyst:} None
\item \textbf{Bonus:} None
\item \textbf{Notes:} “Stances” are Free reactions committed at the beginning of the \emph{counterbeat}, which take effect for that entire \emph{counterbeat}. They do not prevent the character from committing a reaction during the first enemy Turn
\end{itemize}

\begin{center}
\begin{tabularx}{\textwidth}{p{0.1\textwidth}p{0.1\textwidth}p{0.05\textwidth}p{0.05\textwidth}p{0.1\textwidth}p{0.46\textwidth}}
\hline
\rowcolor{white} \textbf{Name} & \textbf{Req} & \textbf{Int} & \textbf{FP} & \textbf{SP} & \textbf{Effects}\setcounter{rownum}{0}\\
\hline
\makeitem{Leaf Stance} & DEX: 12 & 1 & 3 & F & \emph{Reaction.} May move in any direction when suffering Knockback this \emph{counterbeat}. Activating this attunement is done at the start of the \emph{counterbeat}, and does not use the reaction window for any attacks\\
\makeitem{Leap} & DEX: 12\newline ENC<EQP & 1 & 2 & 2 & Move 2. Ignore any enemies, hazards, or half-cover obstacles while moving.\newline This is a Simple action\\
\makeitem{Slink} & DEX: 14 & 1 & 4 & Wep*2 & Functions the same as Backstab \\
\makeitem{Warcry} & - & 1 & 3 & F & Re-roll a single \textbf{SP} Die. Remember that re-rolls are always once-per-die and the second result is final\\
\hline
\end{tabularx}
\end{center}